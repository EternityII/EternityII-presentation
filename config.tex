\usepackage[utf8]{inputenc} 
\usepackage[T1]{fontenc}
\usepackage{setspace}
\usepackage[french]{babel}
\usepackage{csquotes}

\usepackage[mode=buildnew]{standalone}

\usepackage{xparse}
%\usepackage{pgfpages}
%\setbeameroption{show notes}
%\setbeameroption{show notes on second screen=right}

\newcounter{sp}


\newcommand{\inseqa}[1]{%
	\rlap{$\displaystyle#1$}%
	\pheqa}

\newcommand*{\pheqa}{
	\phantom{\sum_{\{x,y\} \notin U_1}\Big(d^+_{C_1}(x)+d^+_{C_1}(y)\Big)}}

% footnote sans reférence
%\makeatletter
%\def\blfootnote{\gdef\@thefnmark{}\@footnotetext}
%\makeatother


%\setbeamertemplate{footnote}{%
%\hangpara{2em}{1}%
%\makebox[2em][l]{\insertfootnotemark}\footnotesize\insertfootnotetext\par%
%}

%%%%%%%%%%%%%%%%%%%%%%%%%%%%%%%%%%%%%%%%%%%%%%%%%%%%%%%%%%%%%%%%%%%%%%%%%%%
%%%% ASPECT RATIO %%%%%%%%%%%%%%%%%%%%%%%%%%%%%%%%%%%%%%%%%%%%%%%%%%%%%%%%%
%%%%%%%%%%%%%%%%%%%%%%%%%%%%%%%%%%%%%%%%%%%%%%%%%%%%%%%%%%%%%%%%%%%%%%%%%%%

\ExplSyntaxOn
\NewDocumentCommand{\ifaspectratio}{mmm}
{
	% Recover the option from those passed to the class
	\keys_set:nf { zunbeltz/beameroptions } { \use:c { opt@beamer.cls } }
	\str_if_eq:nVTF { #1 } \l_zunbeltz_aspectratio_tl { #2 } { #3 }
}

% We need to define only one key, the other are treated as `unknown'
\keys_define:nn { zunbeltz/beameroptions }
{
	aspectratio .tl_set:N = \l_zunbeltz_aspectratio_tl,
	aspectratio .initial:n = 43,
	unknown .code:n = {},
}

% Generate the variants we need
\cs_generate_variant:Nn \keys_set:nn { nf }
\cs_generate_variant:Nn \str_if_eq:nnTF { nV }
\ExplSyntaxOff


%%%%%%%%%%%%%%%%%%%%%%%%%%%%%%%%%%%%%%%%%%%%%%%%%%%%%%%%%%%%%%%%%%%%%%%%%%%
%%%%%%%%%%%%%%%%%%%%%%%%%%%%%%%%%%%%%%%%%%%%%%%%%%%%%%%%%%%%%%%%%%%%%%%%%%%
%%%%%%%%%%%%%%%%%%%%%%%%%%%%%%%%%%%%%%%%%%%%%%%%%%%%%%%%%%%%%%%%%%%%%%%%%%%

%%%%%%%%%%%%%%%%%%%%%%%%%%%%%%%%%%%%%%%%%%%%%%%%%%%%%%%%%%%%%%%%%%%%%%%%%%%
%%%% TIKZ %%%%%%%%%%%%%%%%%%%%%%%%%%%%%%%%%%%%%%%%%%%%%%%%%%%%%%%%%%%%%%%%%
%%%%%%%%%%%%%%%%%%%%%%%%%%%%%%%%%%%%%%%%%%%%%%%%%%%%%%%%%%%%%%%%%%%%%%%%%%%

\usepackage{tikz}


%%%%%%%%%%%%%%%%%%%%%%%%%%%%%%%%%%%%%%%%%%%%%%%%%%%%%%%%%%%%%%%%%%%%%%%%%%%
%%%%%%%%%%%%%%%%%%%%%%%%%%%%%%%%%%%%%%%%%%%%%%%%%%%%%%%%%%%%%%%%%%%%%%%%%%%
%%%%%%%%%%%%%%%%%%%%%%%%%%%%%%%%%%%%%%%%%%%%%%%%%%%%%%%%%%%%%%%%%%%%%%%%%%%




%% COLORS
\definecolor{pum}{HTML}{FE5960}
\definecolor{mblack}{HTML}{555554}
\definecolor{jlirmm}{HTML}{EB940B}
\definecolor{blirmm}{HTML}{1273B6}
\definecolor{rlirmm}{HTML}{E3172C}
\definecolor{vlirmm}{HTML}{4FA129}


\colorlet{mcolor}{blirmm}
%\colorlet{mblack}{black!75}


%% TEMPLATE
\usetheme{Luebeck}
\setbeamertemplate{blocks}[shadow=false] 

%\definecolor{beamer@blendedblue}{HTML}{FE5960} % changed this
\colorlet{beamer@blendedblue}{mcolor}

\setbeamercolor{normal text}{fg=black,bg=white}
\setbeamercolor{alerted text}{fg=red}
\setbeamercolor{example text}{fg=green!50!black}

%% CHANGE LA COULEUR PRINCIPALE
\setbeamercolor{structure}{fg=beamer@blendedblue}

%% CHANGE LA COULEUR DU CADRE OU IL Y A LE PLAN
\setbeamercolor{section in head/foot}{bg=mblack}
%\setbeamercolor{author in head/foot}{bg=Brown}
%\setbeamercolor{date in head/foot}{fg=Brown}

\newcommand\coloreditem[1]{
	\item[\textcolor{#1}{\usebeamertemplate{itemize \beameritemnestingprefix item}}]
}
%\newcommand\coloreditem[1]{\item[\tikz\draw[black, fill=#1] (0,0) rectangle (0.15,0.15);]}
%\setbeamertemplate{itemize items}{\tikz\draw[black, fill=mcolor] (0,0) rectangle (0.15,0.15);}

\newcommand*{\mcolbox}[1]{
	\colorbox{mcolor}{\textcolor{white}{#1}}
}
\newcommand*{\Mcolbox}[2]{
	\colorbox{#2}{\textcolor{white}{#1}}
}

\newcommand*{\lieu}[1]{
	{\scriptsize \bsc{\textcolor{mblack}{#1}}}
}

\newcommand*{\MyArc}[1]{
	postaction={decorate,decoration={markings, mark=at position #1 with {\arrow[>=stealth]{>}}}}
}

\newcommand*{\so}[1]{
	\tikz{\draw[thick, #1] (0,0.2) -- (0,0); \draw[->, thick, \chicol](-0.01,0) -- (0.3,0);}
}

\makeatother

%% NEW BLOCKS
\newenvironment{Mblock}[1]{%
	\setbeamercolor{block body}{bg=mcolor!15}
	\setbeamercolor{block title}{bg=mcolor,fg=white}
	\begin{block}{#1}}{\end{block}}

\newenvironment{Vblock}[1]{%
	\setbeamercolor{block body}{bg=vlirmm!15}
	\setbeamercolor{block title}{bg=vlirmm,fg=white}
	\begin{block}{#1}}{\end{block}}

\newenvironment{Rblock}[1]{%
	\setbeamercolor{block body}{bg=rlirmm!15}
	\setbeamercolor{block title}{bg=rlirmm,fg=white}
	\begin{block}{#1}}{\end{block}}


\newenvironment{Jblock}[1]{%
	\setbeamercolor{block body}{bg=jlirmm!15}
	\setbeamercolor{block title}{bg=jlirmm,fg=white}
	\begin{block}{#1}}{\end{block}}

\newenvironment{Bblock}[1]{%
	\setbeamercolor{block body}{bg=blirmm!15}
	\setbeamercolor{block title}{bg=blirmm,fg=white}
	\begin{block}{#1}}{\end{block}}


%%% FRAMES NON NUMEROTEES

\newcommand{\backupbegin}{
	\newcounter{framenumberappendix}
	\setcounter{framenumberappendix}{\value{framenumber}}
	%\setbeamertemplate{footline}{}
	
}
\newcommand{\backupend}{
	\addtocounter{framenumberappendix}{-\value{framenumber}}
	\addtocounter{framenumber}{\value{framenumberappendix}} 
	%\setbeamertemplate{footline}{\vspace{-1cm}\small{\insertframenumber/\inserttotalframenumber}}
}
%\beamertemplatenavigationsymbolsempty
%\newcounter{currentslide}
%\makeatletter
%\setcounter{currentslide}{\the\beamer@slideinframe}
%\makeatother

%%%%%%%%%%%%%%%


%% AJOUTER LE NUMERO DE SLIDE
\addtobeamertemplate{navigation symbols}{}{%
	
	
	\usebeamerfont{footline}%
	\usebeamercolor[fg]{footline}%
	
	\setcounter{sp}{\insertpagenumber}
	\addtocounter{sp}{-\insertframestartpage}
	\addtocounter{sp}{1}
	
	%\hspace{0.95cm}
	%\arabic{sp} %% WARNING : A ENLEVER
	\textcolor{mblack!35}{[}\insertframenumber/\inserttotalframenumber \textcolor{mblack!35}{]}
}



\setbeamerfont{frametitle}{size=\normalsize}